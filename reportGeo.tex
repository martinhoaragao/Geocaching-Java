\documentclass[a4paper,12pt]{article}
\usepackage[portuguese]{babel}
\usepackage{graphicx}
%\graphicspath{ {/home/jessica/Documentos/Engineer/Geocaching-Java} }
\RequirePackage[T1]{fontenc}
\RequirePackage[utf8]{inputenc}

\renewcommand{\baselinestretch}{2}
%\author{no realizado por :P}
\title{Geocaching POO}
\date{Junho 2015}
\begin{document}
\maketitle
\begin{center}
Realizado por: \\
Jéssica Pereira a71164	\\
\includegraphics[height=3\baselineskip,natwidth=369,natheight=430]{jessica0.jpg}\\
Adelino Costa a70563\\
\includegraphics[height=3\baselineskip,natwidth=369,natheight=430]{adelino.jpg}\\
Martinho Aragão a72205 \\
\includegraphics[height=3\baselineskip,natwidth=369,natheight=430]{martinho.jpg}
\end{center}

\pagebreak


\tableofcontents

\pagebreak

\section{Introdução}

\quad
Este trabalho é sobre o conceito de Geocaching conhecido nas redes sociais: pretendemos simular e registar atividades e descobrimentos de caches. Para isso é necessário modularizar e fazer as devidas abstrações na preparação para o trabalho, pois quanto mais abstrações forem criadas mais independentes os módulos serão, podendo depois usar a composição entre estas classes e fornecendo uma melhor compreensão do código e tratamento da informação.
\par
É necessário o estudo e concepção das classes necessárias, estruturas de dados, métodos respetivos, variáveis de instância e de classe necessárias, imaginar a dependência entre as classes (composição) e definir uma hierarquia (nomeadamente criando uma super classe para as Caches).
\par
Para além de criar classes abstratas, também é necessário guardar todos os dados relativos às Caches e aos Utilizadores para poder criar métodos eficientes relativos ao tratamento destes.
\par De entre os requisitos básicos também serão implementados os Eventos, com a simulação da meteorologia e cálculo de distâncias entre caches dado duas coordenadas (coordenadas iniciais e as coordenadas da cache). Estes dois últimos pontos serão implementados também em toda a descoberta de Caches/Atividades e não somente no decorrer de Eventos pois serão úteis nomeadamente para o cálculo de pontuações.

\pagebreak

%Na introduçao foram ditos os objetivos. :P

\section{Modularização - Classes criadas}

%adicionar imagem do trello final do diagrama

\pagebreak
\section{Bases de dados}

\subsection{CacheBase}
\subsection{UserBase}


\pagebreak
\section{Programa Principal e Main}
\quad Na classe GeocachingPOO estão as chamadas de todos os métodos existentes em todas as classes que permitem uma interação a nivel de Objetos.

\par No Main estão todos os Menus e toda a interação I/O entre o utilizador e o nosso programa.Com isto permitimos que o nosso programa possa ser implementado com outras Interfaces Gráficas, nomeadamente para a web, etc. ... 

\pagebreak
\section{Tratamento de Excepções}
\quad Decidimos colocar todas as excepções num package chamado Exceptions para uma melhor visualização das classes e clareza no diagrama.
Para todas as classes que implementam Excepções 
fazemos import deste Package e passamos a excepção entre as classes \em throw. Apenas no main é que fazemos o \em catch para poder imprimir a devida mensagem de erro criada.

\pagebreak
\section{Conclusão}



\pagebreak
\section{}
\subsection{}
\subsubsection{}

\paragraph{paragraph title here}


\end{document}