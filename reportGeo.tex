\documentclass[a4paper,12pt]{article}
\usepackage[portuguese]{babel}
\RequirePackage[T1]{fontenc}
\RequirePackage[utf8]{inputenc}

\renewcommand{\baselinestretch}{2}
%\author{Martinho Aragão numero, Jéssica, João numero}
\title{Geocaching POO}

\begin{document}
\maketitle
\\
\\Realizado por:
\\Martinho Aragão a72205 
\\Jéssica Pereira a71164	
\\Adelino Costa a70563
\\


\tableofcontents

\section{Introdução}
O presente trabalho é sobre o conceito de Geocaching conhecido nas redes sociais: pretendemos simular e registar atividades e descobrimentos de caches. Para isso é necessário a modularização 

Os objetivos são

Está organizado 

Metodologia utilizada

\section{Conclusão}


\section{}
\subsection{}
\subsubsection{}

\paragraph{paragraph title here}


\end{document}